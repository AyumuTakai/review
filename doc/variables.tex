% TeXデータ吐き出しの検討ファイル (ver.2)
\rdreviewversion{2.0}% review_version
\rdbooktitle{Re:VIEWサンプル書籍}% 書籍名
% ルビは難しいなぁ。別の定義が必要そう

\begin{rdnames}{aut}% 著者名
\item {青木峰郎\rdfileas{あお|き|みね|ろう}}
\item {武藤健志\rdfileas{む|とう|けん|し}}
\end{rdnames}

\begin{rdnames}{trl}% 訳者名
\item {角征典\rdfileas{かど|まさ|のり}}
\end{rdnames}

\begin{rdnames}{edt}% 編集者名
\item {高橋征義\rdfileas{たか|はし|まさ|よし}}
\end{rdnames}

% csl: 監修者名
% dsr: デザイナ名
% ill: イラストレータ名

\rdprt{印刷所名}
\rdpbl{出版社名}
% \rdpbl{出版社名\rdfileas{シュッパンシャメイ}} かrdnamesを使いたいケースはある?

\rddate{2017-11-5}% 刊行日。大扉や奥付で使うことがある
% ↓奥付のための履歴情報。実際には配列の配列
\begin{rdhistory}
\begin{rdedition}
\item {2016年1月11日} % 初版1刷
\item {2016年2月14日} % 初版2刷
\end{rdedition}
\begin{rdedition}
\item {2016年12月24日} % 2版1刷
\item {2017年11月5日} % 2版2刷
\end{rdedition}
\end{rdhistory}
% いわゆるiiページに置きたいもの。TeXだと今は使ってないが…
\begin{rdrights}
権利表記文字列(TeX生?)
\end{rdrights}

\rdisbn{ISBN}% 指定があったら奥付に入れたい
\rdtoclevel{3}% 目次掲載tocdepthの値+1相当。1=部および章のみ、2=節まで
\rdsecnolevel{2}% これは渡さなくてもいいかも…。どの見出しレベルまで採番を付けるか。latexbuilderで*を付ける付けないを判断している
\rdtoc{true}% \tableofcontentsを入れるかどうか。値はなんでもよくて定義ありなしで判断させるのがいい?
\rdcoverpage{true}% 表紙を入れるかどうか
\rdcoverfile{TeXファイルパス}% このファイル内容を表紙のTeX断片としてincludeする
\rdcoverimage{画像ファイルパス}% このファイルを表紙の画像にする (coverfile指定がない場合)
% 本当はfullpagegraphicsではっつけたほうがいいんだろうか…
\rdtitlepage{true}% 大扉を入れるか
\rdtitlefile{TeXファイルパス}% このファイル内容を大扉のTeX断片としてincludeする
\rdoriginaltitlefile{TeXファイルパス}% このファイル内容を原書大扉のTeX断片としてincludeする
\rdcreditfile{TeXファイルパス}% このファイル内容をクレジットページのTeX断片としてincludeする
%% 表紙→大扉→原書大扉→クレジットページ→PREDEF→目次→CHAPS の順 (オライリーなどでは大扉と原書大扉の間に権利表記まわりが1pあったりするが…それは原書大扉カスタムファイルのほうで対処)
\rdmakeindex}{true}% 索引を入れるかどうか
\rdcolophonpage{true}% 奥付を入れるか。true/false
\rdcolophonfile{TeXファイルパス}% このファイル内容を奥付のTeX断片としてincludeする
\rdadvfile{TeXファイルパス}% % このファイル内容を広告のTeX断片としてincludeする
\rdbackcover{TeXファイルパス}% このファイル内容を裏表紙のTeX断片としてincludeする
%% CHAPS→APPENDIX→POSTDEF→索引→奥付→広告→裏表紙 の順
% 
% imagedir,fontdir,image_ext,font_extはなくていいかな
\rdhighlight{plistings}% 使用するハイライト。定義がなければハイライトしない

% documentclassに与えるべきオプションは、ここで定義せずにconfig.ymlで定義したほうがよいだろうか…わかっていないと難しいものがありそう
% []内はjlreq.clsで定義されているもの
%
% twoside/oneside
% 仕上がり用紙サイズ
% 出力用紙サイズ
% トンボ有無
% 版面表示有無
% ぬりたし領域幅
% 紙(print)/電子(pdf)
% hyperrefのオプション
% 天から版面までのアキ(mmかzh) [head_space]
% ノドから版面までのアキ(mmかzw) [gutter]
% 行あたり文字数(zw) [line_length]
% 行数 [number_of_lines]
% 欧文基本Q、和文基本Q [fontsize, jafontsize]
% 行送り [baselineskip]
% ノンブル/柱の小口からの位置 [headfoot_sidemargin]
% 
% 以下は使用しない?
% 傍注幅 [sidenote_length]
% 行間 [linegap]
% twocolumn時の段間アキ [column_gap]
% 地アキ [foot_space]
% 始め括弧位置 [open_bracket_pos]
% JLREQ警告 [jlreq_notes]

% 以下はgreencherryクラスでよくあるもの
% 章の開始ページ左右位置: page,odd,even
% 部の開始ページ左右位置: page,odd,even
% 目次の開始ページ左右位置: page,odd,even
% 索引の開始ページ左右位置: page,odd,even
% 奥付の開始ページ左右位置: page,odd,even

% includegprahicsのためのboxオプションはどうしよう

% ver.2の定義例ここまで
%-----------------------------------------------

% TeXデータ吐き出しの検討ファイル (ver.1)
% マクロ名\rdはReVIEW-Defineの意で、別にこの名前にこだわるわけではない(どっちにしろ自動生成だし)
\rd{review_version}{2.0}% review_version
% bookname, languageはいらなそう
\rd{booktitle}{Re:VIEWサンプル書籍}
% ↓元はbooktitleのfile-as属性。読みはいらない? 商業誌だと奥付ルビに使うことがある。が、ルビの位置をこれだけから決めることは不可能なので、意味がないかも
\rd{booktitle_fileas}{リビューサンプルショセキ}
\rd{aut}{青木峰郎}
\rd{aut_fileas}{アオキミネロウ}% 読みはいらない? 商業誌だと奥付ルビに使うことがあるが、上記booktitle_fileasと同じ問題
\rd{author}{武藤健志}
\rd{author_fileas}{ムトウケンシ}
% ↑FIXME:同じパラメータに複数の値が入るときは、どう渡すのが適切か
\rd{trl}{翻訳者名}
% 以下↓は奥付などで使うかなぁ…ほかにもいろいろdoc/config.yml.sample参照。同じパラメータに複数の値の可能性あり
\rd{edt}{編集者名}
\rd{csl}{監修者名}
\rd{dsr}{デザイナ名}
\rd{ill}{イラストレータ名}
% ↑ここまで
\rd{prt}{印刷所名}
\rd{pbl}{出版社名}
\rd{pbl_fileas}{シュッパンシャメイ}
\rd{date}{2017-11-5}% 刊行日。大扉や奥付で使うことがある
% ↓奥付のための履歴情報。実際には配列の配列で、
%   [["初版第1刷の日付", "初版第2刷の日付"], ["第2版第1刷の日付"]]
%  historyが空ならdateを初版1刷に使う。奥付テンプレートと絡んでくる
%  TeXの変数定義で配列の配列 をなんとかするには…
\rd{history}{2016-04-20}% FIXME
% いわゆるiiページに置きたいもの。TeXだと今は使ってないが…
\rd{rights}{権利表記文字列}
\rd{isbn}{ISBN}% 指定があったら奥付に入れたい
\rd{toclevel}{3}% 目次掲載tocdepthの値+1相当。1=部および章のみ、2=節まで
\rd{secnolevel}{2}% これは渡さなくてもいいかも…。どの見出しレベルまで採番を付けるか。latexbuilderで*を付ける付けないを判断している
\rd{toc}{true}% \tableofcontentsを入れるかどうか。FIXME: true/falseを指定するならどうするのが適正?
\rd{coverpage}{true}% 表紙を入れるか。true/false (config.ymlのcover:はちょっとやり方がまずかったのでTeX変数での名前は変えた。)
\rd{coverfile}{TeXファイルパス}% このファイル内容を表紙のTeX断片としてincludeする
\rd{coverimage}{画像ファイルパス}% このファイルを表紙の画像にする (coverfile指定がない場合)
% 本当はfullpagegraphicsではっつけたほうがいいんだろうか…
\rd{titlepage}{true}% 大扉を入れるか。true/false
\rd{titlefile}{TeXファイルパス}% このファイル内容を大扉のTeX断片としてincludeする
\rd{originaltitlefile}{TeXファイルパス}% このファイル内容を原書大扉のTeX断片としてincludeする
\rd{creditfile}{TeXファイルパス}% このファイル内容をクレジットページのTeX断片としてincludeする
%% 表紙→大扉→原書大扉→クレジットページ→PREDEF→目次→CHAPS の順 (オライリーなどでは大扉と原書大扉の間に権利表記まわりが1pあったりするが…それは原書大扉カスタムファイルのほうで対処)
\rd{makeindex}{true}% 索引を入れるか
\rd{colophonpage}{true}% 奥付を入れるか。true/false (config.ymlのcolophon:はちょっとやり方がまずかったのでTeX変数での名前は変えた。)
\rd{colophonfile}{TeXファイルパス}% このファイル内容を奥付のTeX断片としてincludeする
\rd{advfile}{TeXファイルパス}% % このファイル内容を広告のTeX断片としてincludeする
\rd{backcover}{TeXファイルパス}% このファイル内容を裏表紙のTeX断片としてincludeする
%% CHAPS→APPENDIX→POSTDEF→索引→奥付→広告→裏表紙 の順
% 
% imagedir,fontdir,image_ext,font_extはなくていいかな
\rd{highlight}{plistings}% 使用するハイライト。定義がなければハイライトしない
\rd{page_metric}{用紙サイズ または行数などの配列}% geometryのものなので、jlreqベースだとこれじゃない別のもののほうがよさそう
\rd{documentclass}{ドキュメントクラス名}
\rd{documentclass_options}{オプション,オプション}% このへんはバラしたほうがいい?
% ↓たぶんこういうのがいるかな的なものだと
\rd{仕上がり用紙サイズ}{A5とか182mmx233mmとか}
\rd{出力用紙サイズ}{A4とかB5Jとか}
\rd{トンボ有無}{true/false}
\rd{版面表示有無}{true/false}
\rd{ぬりたし領域}{3mm}
\rd{紙か電子か}{print or pdf}
\rd{hyperrefのオプション}{...}% でもconfig.ymlからは指定しない?
\rd{天}{24mm}
\rd{行}{33}
\rd{ノド}{17mm}
\rd{字詰め}{38}
%↑ とやるには、ベースのQと行送りを任意指定? とすると全体のレイアウト調整にかかる?
% 天・地・ノド・小口で指定するほうがいい、という人も多そう
\rd{章の開始ページ左右位置}{page,odd,even}% 値名は適当
\rd{部の開始ページ左右位置}{page,odd,even}
\rd{目次の開始ページ左右位置}{page,odd,even}
\rd{索引の開始ページ左右位置}{page,odd,even}
\rd{奥付の開始ページ左右位置}{page,odd,even}
\rd{includegprahicsのためのboxオプションはどうしよう}{}
\rd{各スタイルの書体(和欧選定はどうする?)、Q数、行送り?}
