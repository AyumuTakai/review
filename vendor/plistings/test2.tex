%#!ptex2pdf -l -e test2.tex
\ifdefined\directlua
  \documentclass{ltjsarticle}
  \usepackage{listings,xcolor}
\else
  \documentclass[uplatex,dvipdfmx]{jsarticle}
  \usepackage{plistings,xcolor}
  \def\zw{zw}
\fi
\usepackage{lmodern}
\usepackage{showexpl}
\lstset{
  language=C, frame=trbl, framesep=5pt,
  commentstyle=\color{green!50!black},
  basicstyle=\ttfamily, basewidth=0.5em,
  keywordstyle=\bfseries\color{blue},
}
\begin{document}
\begin{lstlisting}
#define N 6
#define ARSZ 2*N
#define ARY unsigned int
/* 1 次元配列 ARY data[ARSZ] をテーブルだと思う */
/* ARY は 32bit 符号なし整数と仮定.すると各行 1変数で済む */

/* 読みだし */
inline ARY read(const ARY data[], const int row, const int col) {
  return data[row] & ((ARY)1)<<col;
}
/* 書き込み */
inline void mark(ARY data[], const int row, const int col) {
  data[row] |= ((ARY)1)<<col;
}

unsigned long long counter;
/* 「次に探索」 */
int next_pos_x, next_pos_y;
/* pos_history[i][k] は,i 番の短冊の (k-1) 番目の cellの位置を
   pack(row, col) = (row << 8) + col という 16ビット整数として管理する.*/
int pos_history[(N*N)<<2];
#define pack(x,y) ((x)<<8)+(y) 
\end{lstlisting}

\lstinputlisting[language={[LaTeX]TeX}]{\jobname.tex}

{\catcode`\^^@=12 \global\let\NUL=^^@
\begin{LTXexample}
 \西暦\today,\和暦\today
 \par{\ttfamily\meaning\西暦}
 a\NUL a^^@a
\end{LTXexample}}
なお,\lstinline|^^@|~(\lstinline|\0|) を陽にソース中に書いた場合は
エラーが発生する.
\end{document}
